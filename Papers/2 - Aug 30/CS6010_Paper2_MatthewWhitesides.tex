% options that should be set.
\documentclass[journal,onecolumn]{IEEEtran}

% correct bad hyphenation here
\hyphenation{op-tical net-works semi-conduc-tor}

\begin{document}

%
% paper title
% Titles are generally capitalized except for words such as a, an, and, as,
% at, but, by, for, in, nor, of, on, or, the, to and up, which are usually
% not capitalized unless they are the first or last word of the title.
% Linebreaks \\ can be used within to get better formatting as desired.
% Do not put math or special symbols in the title.
\title{Seminar Talk: ``Computational Sustainability: Computing for a Better World and a Sustainable Future'' (Speaker: Carla Gomes)}

%
%
% author names and IEEE memberships
% note positions of commas and nonbreaking spaces ( ~ ) LaTeX will not break
% a structure at a ~ so this keeps an author's name from being broken across
% two lines.
% use \thanks{} to gain access to the first footnote area
% a separate \thanks must be used for each paragraph as LaTeX2e's \thanks
% was not built to handle multiple paragraphs
%
\author{Matthew~Whitesides}% <-this % stops a space

% The paper headers
\markboth{Missouri S\&T COMP\_SCI 6010: Seminar, Fall~2021}%
{Shell \MakeLowercase{\textit{et al.}}: Bare Demo of IEEEtran.cls for IEEE Journals}

% make the title area
\maketitle 

% As a general rule, do not put math, special symbols or citations
% in the abstract or keywords.
\begin{abstract}
  Computer and information scientists join forces with other fields to help solve societal and environmental challenges facing humanity, in pursuit of a sustainable future. 
  Using computational research into various aspects of sustainability Carala Gomes and peers survey research topics ranging from biodiveristy, environmental conservation, renewable energy, sustainable materials, and animal species conservation.   
\end{abstract}

% Note that keywords are not normally used for peerreview papers.
% \begin{IEEEkeywords}
% IEEE, IEEEtran, journal, \LaTeX, paper, template.
% \end{IEEEkeywords}

\IEEEpeerreviewmaketitle

\section{Introduction}

\IEEEPARstart{E}{v}ery day great leaps are taking place in computational technology, from machine learning to artificial intelligence, to cloud computing and big data, the amount of progress is seemingly unlimited. 
These advances are not only applied in a research sense but have a practical everyday impact on people. "Smart" technology, cities, transportation, communication all directly consumes these advances and impact lives almost as soon as the technology is developed. 
However, despite all this technological advancement, nearly a billion people live below the poverty line, and climate change threatens our planet.
Unfortunately, given the global human issues, the primary uses of these technologies are first and foremost go-to sectors that can make the most profit from them, i.e., large corporations and consumer technology. 
Much less talent and resources are devoted to solving these fundamental human issues in preference for profit. 
In today's presentation, Professor Carla Gomes discusses the challenges facing our modern world and how computing can be used to help build a sustainable future for everyone. 

\section{Background}

Computational sustainability is a relatively new field of research that aims to identify, formalize, and provide solutions to computational problems concerning balancing environmental, economic, and societal needs for a sustainable future. [1] 
The research field of computational sustainability is directly tied to AI, machine learning (ML), data science, and computer science in general.
Using computer science, we can tackle the relatively new field of computational sustainability to help solve some of the critical challenges concerning environmental, economic, and social issues to ensure a hopeful future.

\section{Research Contributions and Results}

There are three main areas of focus for computational sustainability and sustainable development, balancing environmental and socioeconomic needs, biodiversity and conservation; and, renewable and sustainable energy and materials. 
Let us look at what each of these significant issues entails, and an example of current exciting research professor Gomes and the team are working on in each field.

\subsection{Balancing Environmental and Socioeconomic Needs}

Improving the socioeconomic placement of people is an ongoing human challenge. Over 800 million people live below the international poverty line of 1.90 dollars per day. 
Not only that, but the situation is likely only to become more difficult to fix over time as resources are consumed and need to be better conserved balanced with the rapid overall population growth. 
Various quality datasets need to be produced to identify socioeconomic issues, which is a challenge within itself. Poverty mapping is complicated as inherently, and the most impoverished areas are the areas where the least amount of data exists and is most difficult to track accurately. 

To approach the challenges of gaining insight into impoverished areas, we can use various computational methods. 
Machine learning can identify markers of poverty that are not immediately apparent and indicate significant issues to tackle first. 
Using these techniques expands beyond identifying areas in need but real solutions as well. 
For example, there are new advances in maximizing HIV prevention and treatment using AI, which is another issue that disproportionately affects the lower socioeconomic statuses. [3]
Topics as important as socioeconomic needs also must consider making fair and ethical decisions. Therefore any AI or ML model must consider multiple criteria instead of optimizing to a single solution that might become biased. 

\subsection{Biodiversity and Conservation}

Due to agriculture, urbanization, deforestation, and other human-driven destruction of natural environments, biodiversity continues to shrink and threatens the planet and all species that live on it, including humanity.
When boiling biodiversity down to a data problem, the fundamental challenge involves understanding how different species are distributed across landscapes over time, which creates spatial and temporal modeling and prediction problems to be researched.
Finally, climate change itself is another central area of research impacting environmental conservation on a global scale. Utilizing intelligent models and methods is vital due to the international and massive amount of data potentially generated by weather and ecological datasets. 

Professor Gomes mentions one project involving a large-scale corporation between the Cornell Lab of Ornithology's data scientists, statisticians, and the global birding community to build the eBird dataset. 
This project's idea is simple, to work with the existing and vast birdwatching community worldwide to track populations of various species across the globe and create a significant detailed birding dataset. 
The eBird dataset contains an endless possibility of aiding bird conservation, including various statistical, machine learning models, and AI to estimate population movement and count and identify areas in danger. 

\subsection{Renewable and Sustainable Energy and Materials}

Last but not least is in the area of renewable and sustainable energy. 
Inherently sustainability from a data science perspective is an optimization problem. 
Electricity management for cities is a complex problem, and adding renewable energy methods adds to the complexity. 
Generating, distributing, and storing excess energy is a complex and expensive task that AI and ML can solve. 
For example, SMART-Invest is a data model that aims to optimize investments in wind, solar, and storage in real-world scenarios over time. 

Another area of research involves material sciences and processes. New materials and techniques aim to create efficient energy storage such as batteries, fuel cells, solar fuels, microbial fuel cells.
Furthermore, efficient processes aid in C02 reduction, or models can be optimized to minimize these impacts.

One example that shows the challenges faced on every layer is in the Amazon hydropower dam planning project. 
Hydropower itself is a renewable energy source. However, building infrastructure in the Amazon is a disruptive environmental event. 
There is a statistical trade-off between impact and benefit that must be considered in the project's planning phase, and for these to be accurate, complex dynamic algorithms are used to model the environmental impact of different situations.

\section{Lessons Learned}

Computational sustainability is an excellent topic that, unfortunately, is not discussed enough in forward-looking research. As with most things, if there's not an immediate profitable use case for the technology, it does not get as much attention. 
However, as professor Gomes points out, in the long run, a lot of these problems will lead to continued environmental and economic sustainability, which benefits everyone on the planet. 
Furthermore, in hindsight, showing how these issues are broken down or benefit from visualizing them as data or computational problems seems obvious. 
Due to the global scale of these issues, big data and AI techniques are well suited to solving these issues. 
These techniques are crucial to identifying issues, optimizing solutions, and discovering new methods to tackle the most significant problems facing the world today. 

\section{Conclusion}

In the end, the overall goal of sustainable development is to ensure the well-being of current and future generations, a noble goal worth fighting for.
Hopefully, this can go hand in hand with the significant advances in computational technology. 
Advances in the internet of things and connected devices apply to environmental issues that utilize large-scale sensor data. 
Biometric adoption can lead to comprehensive societal datasets that can help health organizations identify the most at-risk populations. 
Energy storage advancements have become a hugely profitable area in mobile computing and electric transportation, which is vital to the future of sustainable energy. 
These are just a few examples of ways profit and computational sustainability can work together, and hopefully, work from people like Carla Gomes and colleagues is only the beginning. 

% \appendices
% \section{Proof of the First Zonklar Equation}
% Appendix one text goes here.

% % you can choose not to have a title for an appendix
% % if you want by leaving the argument blank
% \section{}
% Appendix two text goes here.


% use section* for acknowledgment
\section*{Acknowledgment}
The author would like to thank Professor Sajal Das with the Department of Computer Science, Missouri University of Science and Technology and Professor Carla Gomes with Cornell University.

% Can use something like this to put references on a page
% by themselves when using endfloat and the captionsoff option.
\ifCLASSOPTIONcaptionsoff
  \newpage
\fi

\begin{thebibliography}{1}

\bibitem{IEEEhowto:Gomes}
C.P.~Gomes, ``Computational sustainability: Computational methods for a sustainable environment."

\bibitem{IEEEhowto:Gomess}
C.P.~Gomes, et al., ``Computational sustainability: computing for a better world and a sustainable future.", Commun. ACM 62, 9 (September 2019), 56–65. DOI:https://doi.org/10.1145/3339399

\bibitem{IEEEhowto:Gomess}
M.~Tambe, and E.~Rice, ``Artificial Intelligence and Social Work.", Cambridge University Press, 2018

\end{thebibliography}

% biography section
\begin{IEEEbiographynophoto}{Matthew Whitesides}
  Master's Student at Missouri University of Science and Technology.
\end{IEEEbiographynophoto}

% that's all folks
\end{document}