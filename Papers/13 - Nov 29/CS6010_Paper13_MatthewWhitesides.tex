% options that should be set.
\documentclass[journal,onecolumn]{IEEEtran}

% correct bad hyphenation here
\hyphenation{op-tical net-works semi-conduc-tor}

\begin{document}

%
% paper title
% Titles are generally capitalized except for words such as a, an, and, as,
% at, but, by, for, in, nor, of, on, or, the, to and up, which are usually
% not capitalized unless they are the first or last word of the title.
% Linebreaks \\ can be used within to get better formatting as desired.
% Do not put math or special symbols in the title.
\title{Seminar Talk: ``Data Mining for the Relationship Among Consciousness, AI, and Coordination'' (Speaker: Prof. Ouri Wolfson)}

%
%
% author names and IEEE memberships
% note positions of commas and nonbreaking spaces ( ~ ) LaTeX will not break
% a structure at a ~ so this keeps an author's name from being broken across
% two lines.
% use \thanks{} to gain access to the first footnote area
% a separate \thanks must be used for each paragraph as LaTeX2e's \thanks
% was not built to handle multiple paragraphs
%
\author{Matthew~Whitesides}% <-this % stops a space

% The paper headers
\markboth{Missouri S\&T COMP\_SCI 6010: Seminar, Fall~2021}%
{Shell \MakeLowercase{\textit{et al.}}: Bare Demo of IEEEtran.cls for IEEE Journals}

% make the title area
\maketitle

% As a general rule, do not put math, special symbols or citations
% in the abstract or keywords.
\begin{abstract}
  In today's presentation, Professor Ouri Wolfson takes us through a fascinating introduction to the philosophical and practical implications of modeling human-level intelligence into a software system. Many works suggest that consciousness cannot be achieved in software, and many believe it can. The benefits to this work beyond the simple achievement can apply to making more rational and empathetic decisions in a world that increasingly relies on computer decision-making processes. 
\end{abstract}

% Note that keywords are not normally used for peerreview papers.
% \begin{IEEEkeywords}
% IEEE, IEEEtran, journal, \LaTeX, paper, template.
% \end{IEEEkeywords}

\IEEEpeerreviewmaketitle

\section{Introduction}

\IEEEPARstart{W}{hat} is consciousness? Is this something intrinsic to humans and animals? If we could perfectly model a human brain in a computer system, would that system be considered conscious? Professor Ouri Wolfson tackles that challenge today by surveying the existing research done on the subject and proposing a simple mathematical system that can make a system behave like a conscious entity to make more intelligent and ethical decisions. 

The famous Frank Jackson thought experiment proposes a scientist who has lived their whole life in a black and white room studying the various properties of color, having never actually seen it for themselves. This experiment poses the question that even though she knows everything, there is to know about color (i.e., the computer information). Will anything new be learned when she sees it for herself (i.e., her subjective experience)? This question is a fascinating exercise in the context of modeling a conscious being in a computer; we can only provide it information we cannot give it a subjective experience, which very well may be what consciousness is in and of itself. 

We can simulate scenarios and experiences to truly achieve Human-level AI (HAI) or generalized artificial intelligence. We need it to be conscious, something we can't even fully agree on the definition of, let alone build a conscious system. Some ideas on creating a conscious system include attempting to merge human brains and machines or creating a simulation of the brain in a computer system. Integrating the brain and machine could connect our brains more closely to machines (i.e., neural link interface). To simulate a brain, we could create a symbiotic relationship or fully upload our brains to a computer, thus simulating a brain in a machine that, if done without information loss, would be a conscious being itself. 

These ideas all have fascinating implications for the sake of research itself; however, there are particular practical implications of this type of research. As our security becomes more and more data-driven, one primary concern is the concept of fair and ethical decision-making and bias in data-driven decisions. So many things in our modern world are becoming data-driven, such as social media, financial rates, civil engineering, etc. This strategy is good because more informed and efficient methods are being employed, but it becomes problematic if the models are built with unseen biases or issues. However, a more human-like AI could spot these issues or make decisions in a more ethical human-centric way. 

\section{Background}

The topic of simulated consciousness has been heavily theorized and researched for decades. 

In "A neuronal network model linking subjective reports and objective physiological data during conscious perception," Dehaene S et al. explore the subjective experience of perceiving visual stimuli accompanied by objective neuronal activity patterns. In this paper, the authors model neurons that form workspace relationships with surrounding neurons to process stimuli selectively. This model subjectively models consciousness for a given stimulus due to the subjective neuron coordination group. 

Dennett, D.C. questions this idea of the global neuronal workspace model of consciousness in their paper "Are we explaining consciousness yet?". The Global Workspace Theory (GWT) is a cognitive architecture that accounts qualitatively for many matched pairs of conscious and unconscious processes. This theory explores consciousness by separating the stimulus (observable data) and the subject experience the human or system has going in the background. 

Finally, in A. Demertzi et al.'s "Human consciousness is supported by dynamic complex patterns of brain signal coordination." the authors attempt to specify consciousness by comparing brain signal coordination between responsive and unresponsive patients. Once isolated, in turn, would help prove that this coordination is critical in defining consciousness and how a system could model it.

\section{Research Contributions and Results}

\subsection{Competition/Coordination in Traffic Analysis}

Professor Wolfson proposes a scenario in which you have 3600 vehicles with two routes to choose from. Route one takes one second per vehicle, while route two takes a flat one hour. If we have 3600, each path will end up taking an hour, so if left up to each vehicle to decide for themselves, it seems evident that each will individually choose the one-second route. However, what is best is if each vehicle cooperated and is divided up between the two routes giving a lower overall average to all vehicles even though some will still take the entire hour. 

A system that decides in pure competition between each vehicle would never choose the route that guarantees them an hour travel time. However, one that decides in coordination could optimize this travel and benefit the most significant number of vehicles. This idea of cooperation is key to modeling a conscious system, neural act in groups or as a whole, not individually. It shows how acting in this manner leads to a greater good rather than an individual selfish one. The  

\subsection{Functional Connectome (FC)}

To model a human-like brain in a computer, we need to model individual neurons that make up the brain. These neurons are interconnected with each other and can fire electrical signals between each other. There are over one hundred billion neurons and one trillion synapses between them in the brain. Each neuron can fire roughly five to fifty times a second. A weighted graph can closely model this, a vertex represents each neuron, and the connections between them are the weighted edges.
On top of that, there are regions of nodes that model brain regions and weights between nodes that model the activity of a given region. This area of study is the study of connectomics, which is the production and research of connectomes. Connectomes are the graph relationships of connections within an organism's nervous system.

The functional connections are ultimately person-specific, and beyond that specific, to the particular task the person is performing, i.e., different neural paths are utilized when writing vs. driving a car. This activity is part of an individual's subjective background experience, differentiating between simple input/output and consciousness.

\section{Lessons Learned}

Future ideas to be expanded upon include proving the traffic coordination model is a conscious behavior, determining mechanisms by which coordination can be achieved in the brain, and ultimately how to build a conscious ethical AI. I have often thought about the philosophical ideas proposed here, such as creating a conscious AI, brain and machine interfaces, and modeling a brain in a system. However, the actual practical ideas of modeling consciousness through subjective experience and the global corporation are fascinating. These ideas have real practical applications in making informed ethical decisions that society can utilize in machine learning and AI systems. Regardless is you could consider these systems to be living like an organic conscious being. It's an exciting area of advancement in AI. Typically philosophical thought experiments end at the thought phase but attempting to create valuable models out of these ideas is very cool. I believe techniques such as connectomics and employing more neuron-like activity in our graph models are critical in advancing computational intelligence research. So many advancements in mathematics and science have come from looking to the natural world for inspiration, which is another logical area to research. Plus, who knows, these more straightforward steps, in turn, may lead to actually creating conscious human-like artificial intelligence or allow us to simulate real-life beings sooner than anyone expected. 

\section{Conclusion}

The fundamental philosophical ideas of consciousness have been around since the beginning of human consciousness and likely will be until the end. However, researching areas on how to apply provable mathematical models to consciousness and intelligence can help us understand our consciousness and have practical implications in ethical data-driven decision-making. Prof. Wolfson provides us with an exciting overview of the challenges and possibilities of consciousness modeling and conscious decision-making research. We can make better choices by proposing systems that take the idea of neurons working together as a whole and applying it to practical systems such as traffic grids and efficiency graphs. The idea that a system can adaptively consider the benefits of the whole rather than short-sided upfront efficiencies is key to making ethical, safe, and fair decisions. These choices will impact our world as we grow increasingly reliant on data-based, AI, and machine-learned decisions in our society. If this type of system can be called conscious is up for debate, and likely regardless of how advanced it becomes, it always will be. However, I'm fascinated by the idea of taking this infinitely grand of a philosophical subject and boiling it down to practical mathematical concepts that will ultimately help us understand ourselves and build better software. 

% \appendices
% \section{Proof of the First Zonklar Equation}
% Appendix one text goes here.

% % you can choose not to have a title for an appendix
% % if you want by leaving the argument blank
% \section{}
% Appendix two text goes here.


% use section* for acknowledgment
\section*{Acknowledgment}
The author would like to thank Professor Sajal Das with the Department of Computer Science, Missouri University of Science and Technology and Prof. Ouri Wolfson with the University of Illinois, Chicago.

% Can use something like this to put references on a page
% by themselves when using endfloat and the captionsoff option.
\ifCLASSOPTIONcaptionsoff
  \newpage
\fi

% \begin{thebibliography}{1}

% \bibitem{IEEEhowto:kopka}
% H.~Kopka and P.~W. Daly, \emph{A Guide to \LaTeX}, 3rd~ed.\hskip 1em plus
%   0.5em minus 0.4em\relax Harlow, England: Addison-Wesley, 1999.

% \end{thebibliography}

% biography section
\begin{IEEEbiographynophoto}{Matthew Whitesides}
  Master's Student at Missouri University of Science and Technology.
\end{IEEEbiographynophoto}

% that's all folks
\end{document}