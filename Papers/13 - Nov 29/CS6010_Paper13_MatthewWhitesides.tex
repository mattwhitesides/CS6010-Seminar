% options that should be set.
\documentclass[journal,onecolumn]{IEEEtran}

% correct bad hyphenation here
\hyphenation{op-tical net-works semi-conduc-tor}

\begin{document}

%
% paper title
% Titles are generally capitalized except for words such as a, an, and, as,
% at, but, by, for, in, nor, of, on, or, the, to and up, which are usually
% not capitalized unless they are the first or last word of the title.
% Linebreaks \\ can be used within to get better formatting as desired.
% Do not put math or special symbols in the title.
\title{Seminar Talk: ``From Smart Sensing to Smart Living'' (Speaker: Sajal Das)}

%
%
% author names and IEEE memberships
% note positions of commas and nonbreaking spaces ( ~ ) LaTeX will not break
% a structure at a ~ so this keeps an author's name from being broken across
% two lines.
% use \thanks{} to gain access to the first footnote area
% a separate \thanks must be used for each paragraph as LaTeX2e's \thanks
% was not built to handle multiple paragraphs
%
\author{Matthew~Whitesides}% <-this % stops a space

% The paper headers
\markboth{Missouri S\&T COMP\_SCI 6010: Seminar, Fall~2021}%
{Shell \MakeLowercase{\textit{et al.}}: Bare Demo of IEEEtran.cls for IEEE Journals}

% make the title area
\maketitle

% As a general rule, do not put math, special symbols or citations
% in the abstract or keywords.
\begin{abstract}
  In today's presentation, Dr. Ouri Wolfson takes us through a fascinating introduction to the philosophical and practical implications of modeling human-level intelligence into a software system. Many works suggest that consciousness cannot be achieved in software, and many believe it can. The benefits to this work beyond the simple achievement can apply to making more rational and empathetic decisions in a world that increasingly relies on computer decision-making processes. 
\end{abstract}

% Note that keywords are not normally used for peerreview papers.
% \begin{IEEEkeywords}
% IEEE, IEEEtran, journal, \LaTeX, paper, template.
% \end{IEEEkeywords}

\IEEEpeerreviewmaketitle

\section{Introduction}

\IEEEPARstart{W}{hat} is consciousness? Is this something intrinsic to humans and animals? If we were able to perfectly model a human brain in a computer system would that system be considered conscious? Dr. Ouri Wolfson tackles that challange today by taking a survey of the existing reseach done on the subject and proposing a simple mathematical system that can make a system behave like a conscious entity to make more intelligent and ethical decisions. 



\section{Background}

Lorem ipsum dolor sit amet, consectetur adipiscing elit.

\section{Research Contributions and Results}

\subsection{Functional Connectome (FC)}

To model a human-like brain in a computer, we need to model individual neurons that make up the brain. These neurons are interconnected with each other and can fire electrical signals between each other. A weighted graph most closely models this, a vertex represents each neuron, and the connections between them are the weighted edges. This area of study is the study of connectomics, which is the production and research of connectomes. Connectomes are the graph relationships of connections within an organism's nervous system.

The functional connections are ultimately person-specific, and beyond that specific, to the particular task the person is performing, i.e., different neural paths are utilized when writing vs. driving a car. 

\section{Lessons Learned}

Future ideas to be expanded upon include, proving the traffic coordination model is a conscious behavior, determine mechanisms by which coordination can be achieved in the brain, and ultimately how to build a conscious ethical AI. 

\section{Conclusion}

The fundamental philosophical ideas of consciousness have been around since the beginning of human consciousness and likely will be until the end. However, researching areas on how to apply provable mathematical models to consciousness and intelligence can help us understand our consciousness and have practical implications in ethical data-driven decision-making. Dr. Wolfson provides us with an exciting overview of the challenges and possibilities of consciousness modeling and conscious decision-making research. We can make better choices by proposing systems that take the idea of neurons working together as a whole and applying it to practical systems such as traffic grids and efficiency graphs. The idea that a system can adaptively consider the benefits of the whole rather than short-sided upfront efficiencies is key to making ethical, safe, and fair decisions. These choices will impact our world as we grow increasingly reliant on data-based, AI, and machine-learned decisions in our society. If this type of system can be called conscious is up for debate, and likely regardless of how advanced it becomes, it always will be. However, I'm fascinated by the idea of taking this infinitely grand of a philosophical subject and boiling it down to practical mathematical concepts that will ultimately help us understand ourselves and build better software. 

% \appendices
% \section{Proof of the First Zonklar Equation}
% Appendix one text goes here.

% % you can choose not to have a title for an appendix
% % if you want by leaving the argument blank
% \section{}
% Appendix two text goes here.


% use section* for acknowledgment
\section*{Acknowledgment}
The author would like to thank Professor Sajal Das with the Department of Computer Science, Missouri University of Science and Technology.

% Can use something like this to put references on a page
% by themselves when using endfloat and the captionsoff option.
\ifCLASSOPTIONcaptionsoff
  \newpage
\fi

% \begin{thebibliography}{1}

% \bibitem{IEEEhowto:kopka}
% H.~Kopka and P.~W. Daly, \emph{A Guide to \LaTeX}, 3rd~ed.\hskip 1em plus
%   0.5em minus 0.4em\relax Harlow, England: Addison-Wesley, 1999.

% \end{thebibliography}

% biography section
\begin{IEEEbiographynophoto}{Matthew Whitesides}
  Master's Student at Missouri University of Science and Technology.
\end{IEEEbiographynophoto}

% that's all folks
\end{document}