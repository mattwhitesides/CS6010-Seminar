% options that should be set.
\documentclass[journal,onecolumn]{IEEEtran}

% correct bad hyphenation here
\hyphenation{op-tical net-works semi-conduc-tor}

\begin{document}

%
% paper title
% Titles are generally capitalized except for words such as a, an, and, as,
% at, but, by, for, in, nor, of, on, or, the, to and up, which are usually
% not capitalized unless they are the first or last word of the title.
% Linebreaks \\ can be used within to get better formatting as desired.
% Do not put math or special symbols in the title.
\title{Seminar Talk: ``Impact of Transportation Networks, Vaccines, ...'' (Speaker: Dr. Philip Pare)}

%
%
% author names and IEEE memberships
% note positions of commas and nonbreaking spaces ( ~ ) LaTeX will not break
% a structure at a ~ so this keeps an author's name from being broken across
% two lines.
% use \thanks{} to gain access to the first footnote area
% a separate \thanks must be used for each paragraph as LaTeX2e's \thanks
% was not built to handle multiple paragraphs
%
\author{Matthew~Whitesides}% <-this % stops a space

% The paper headers
\markboth{Missouri S\&T COMP\_SCI 6010: Seminar, Fall~2021}%
{Shell \MakeLowercase{\textit{et al.}}: Bare Demo of IEEEtran.cls for IEEE Journals}

% make the title area
\maketitle

% As a general rule, do not put math, special symbols or citations
% in the abstract or keywords.
\begin{abstract}
  In today's presentation, Dr. Philip Pare and the team take on one of the biggest challenges of our point in history, modeling and controlling how the COVID-19 virus spreads. Dr. Pare focuses on a few key areas, extending the SEIR model to transportation networks across the northeast of the United States, evaluating travel ban mitigation strategies, and modeling vaccine hesitancy vs. the spread. 
\end{abstract}

% Note that keywords are not normally used for peerreview papers.
% \begin{IEEEkeywords}
% IEEE, IEEEtran, journal, \LaTeX, paper, template.
% \end{IEEEkeywords}

\IEEEpeerreviewmaketitle

\section{Introduction and Background}

\IEEEPARstart{C}{ompartmental} models is a general modeling technique often applied to the mathematical modeling of infectious diseases. Specifically, a Susceptible, Exposed, Infectious, Recovered (SEIR) model creates a deterministic model that can simulate an infection throughout the spread phases. Mainly includes parameters for diseases with a latent phase during which the individual is infected but not yet infectious. Dr. Philip Pare and his team have extended a networked SEIR model on the COVID-19 disease to include transportation network data, specifical data for air travel across the United States. Using this model, we will better predict the spread and estimate infection levels of COVID-19 across urban countries. 

\section{Research Contributions}

\subsection{Transportation Network Model}

Transportation network analysis provide new methods to analyze evolution of the population flow and understand its influence on COVID-19 transmission. 

\subsection{Vaccine Hesitancy Model}

\section{Results}

Lorem ipsum dolor sit amet, consectetur adipiscing elit.

\section{Lessons Learned}

Lorem ipsum dolor sit amet, consectetur adipiscing elit.

\section{Conclusion}
The recent COVID-19 outbreak has spawned a massive surge in research into understanding the spread of infectious diseases. Understanding the reach and creating mathematical models to predict the future spread and better ways of controlling spread will be vital to bring the current pandemics to an end and any future ones we may encounter. Dr. Pare and his team continue to aid in this ongoing research with his contributions in modeling transportation networks and vaccine hesitancy against the spread of the COVID-19 disease. 

% \appendices
% \section{Proof of the First Zonklar Equation}
% Appendix one text goes here.

% % you can choose not to have a title for an appendix
% % if you want by leaving the argument blank
% \section{}
% Appendix two text goes here.


% use section* for acknowledgment
\section*{Acknowledgment}
The author would like to thank Professor Sajal Das with the Department of Computer Science, Missouri University of Science and Technology and Dr. Philip Pare with Purdue University.

% Can use something like this to put references on a page
% by themselves when using endfloat and the captionsoff option.
\ifCLASSOPTIONcaptionsoff
  \newpage
\fi

% \begin{thebibliography}{1}

% \bibitem{IEEEhowto:kopka}
% H.~Kopka and P.~W. Daly, \emph{A Guide to \LaTeX}, 3rd~ed.\hskip 1em plus
%   0.5em minus 0.4em\relax Harlow, England: Addison-Wesley, 1999.

% \end{thebibliography}

% biography section
\begin{IEEEbiographynophoto}{Matthew Whitesides}
  Master's Student at Missouri University of Science and Technology.
\end{IEEEbiographynophoto}

% that's all folks
\end{document}