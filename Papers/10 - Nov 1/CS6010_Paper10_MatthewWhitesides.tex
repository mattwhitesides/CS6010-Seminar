% options that should be set.
\documentclass[journal,onecolumn]{IEEEtran}

% correct bad hyphenation here
\hyphenation{op-tical net-works semi-conduc-tor}

\begin{document}

%
% paper title
% Titles are generally capitalized except for words such as a, an, and, as,
% at, but, by, for, in, nor, of, on, or, the, to and up, which are usually
% not capitalized unless they are the first or last word of the title.
% Linebreaks \\ can be used within to get better formatting as desired.
% Do not put math or special symbols in the title.
\title{Seminar Talk: ``Virtual Campus Visit with NSF CISE Assistant Director'' (Speaker: Dr. Margaret Martonosi}

%
%
% author names and IEEE memberships
% note positions of commas and nonbreaking spaces ( ~ ) LaTeX will not break
% a structure at a ~ so this keeps an author's name from being broken across
% two lines.
% use \thanks{} to gain access to the first footnote area
% a separate \thanks must be used for each paragraph as LaTeX2e's \thanks
% was not built to handle multiple paragraphs
%
\author{Matthew~Whitesides}% <-this % stops a space

% The paper headers
\markboth{Missouri S\&T COMP\_SCI 6010: Seminar, Fall~2021}%
{Shell \MakeLowercase{\textit{et al.}}: Bare Demo of IEEEtran.cls for IEEE Journals}

% make the title area
\maketitle

% As a general rule, do not put math, special symbols or citations
% in the abstract or keywords.
\begin{abstract}
  The National Science Foundation (NSF) supports a majority of academic research in the United States for Computer and Information Science and Engineering (CISE). The NSF CISE has a mission statement to improve scientific discovery and engineering innovation through its support of fundamental research and education in computer and information science and engineering and transformative advances in research cyberinfrastructure. Since February 2020, Dr. Margaret Martonosi has served as the Assistant Director of the NSF CISE. In today's presentation, Dr. Martonosi invites us to learn about how ideas are created and research is done at the NSF CISE, taking us through the promising work to solve problems today and in the future. 
\end{abstract}

% Note that keywords are not normally used for peerreview papers.
% \begin{IEEEkeywords}
% IEEE, IEEEtran, journal, \LaTeX, paper, template.
% \end{IEEEkeywords}

\IEEEpeerreviewmaketitle

\section{Introduction and Background}

\IEEEPARstart{I}{nformation} and communication research are vital to our modern world in most of the technology we use in our everyday lives and technologies we have not foreseen yet. Today's research can have a broad and lasting effect on society and solve problems we may not know we had yet. For example, all the work done over the past few decades in networking and image processing technology has enabled billions of people worldwide to keep working together and being produced through a worldwide pandemic. This outcome was not the original goal of this research but an unforeseen massive impact when needed. The NSF CISE has a mission statement to enable the U.S. to uphold its leadership in computing, communications, and information science and engineering. To achieve this mission, the CISE supports investigator-initiated research and education in all areas of computer and information science and engineering. In addition to an Office of the Assistant Director, CISE includes four other research units, the Office of Advanced Cyberinfrastructure (OAC), Division of Computing and Communication Foundations (CCF), Division of Computer and Network Systems (CNS), and Division of Information and Intelligent Systems (IIS).

The NSF CISE was founded after World War II in an effort to combine information and computer science research with the other natural sciences (i.e., biological, engineering, mathematical, etc.). The CISE organization and core programs include:

\begin{itemize}
  \item Advanced Cyberinfrastructure (OAC): This organization focuses on advanced computing in data and software, particularly in the networking and cyber security areas. This research enables leadership in advanced computing and promotes learning in the workforce. 
  \item Computing and Communication Foundations (CCF): They create algorithmic, communication, information, software and hardware, and emerging technology foundational research.
  \item Computer and Network Systems (CNS): The CNS produces advancements in computer network systems and enables workforce education in these advancements. 
  \item Information and Intelligent Systems (IIS): The IIS enables research in computer information such as human cyber systems, information integration and informatics, and robust intelligence systems.
\end{itemize}

In addition to these critical areas of focus, the CISE has various comprehensive and multi-directorate initiatives. Overall the NSF founds more than eighty-five percent of the federally funded academic computer science research in the United States.

\section{Research Contributions and Results}

The CISE has a few forward-looking overall themes in academic research to focus on. 

\subsection{CISE in a Post-Moore World: Seismic Shift}

Without a doubt, technology and application trends are reshaping computing. There are fundamental changes to the systems and applications we are building primarily due to technologies like smartphones, IoT, and cloud computing. Moore predicted an exponential increase in computing technology which has proven true. However, things like multi-core, graphics, and computing efficiencies have broadened the definition of computing advancements beyond that. The slowing of Moore's law is slowing the advancement of transistor building. These fundamental changes in processor developments change how we develop and compile software, requiring software engineering and architecture changes. Beyond this being a challenge, this is also a chance to go back and rethink some of the fundamental decisions made based upon the existing architectural design. NSF has a variety of programs focused on this, including the Foundations of Emerging Technologies, Principles and Practice of Scaleable Systems, and Beyond 5G Advanced Wireless Networks. 

\subsection{Transcendence of Artificial Intelligence}

Artificial intelligence and machine learning have become mainstream in the past few years. These ideas span many themes, and the NSF invests over five hundred million annually in AI-related research. This research includes "core" A.I. technologies such as Robust Intelligence, Information Integration and Informatics, and Cyber-Human Systems. These encapsulate various A.I. initiatives such as autonomous systems, robotics, smart and connected devices, cyber-physical systems, and more. This A.I. research needs to be founded on fairness, accountability, transparency, safety, and security. A lot of this A.I. research is focused on real-world applications such as in the health, agriculture, and infrastructure fields. NSF is the largest non-defense funder of A.I. research, including the national A.I. research institutes and teaming up with company partnerships such as Intel, Google, Amazon, IBM, and NIST. 

\subsection{Sociotechnical Frontier}

CISE's sociotechnical frontier is the connection between society and technology. Systems such as cyber-physical and cyber-human are ever increasing in our community and economy. These ideas cover a vast area of research from wireless connectivity, smart devices and infrastructure, robotics, natural language processing, and verification interfaces. Research in this area can help reduce disparities in health, education, and technological biases. Much like A.I., this research must keep in mind the trustworthiness of the information and empower the diversity in education and the workforce. 

The CIVIC innovation challenge award is to offer better mobility options to solve the spatial mismatch between affordable housing and jobs. A key aspect to this includes the availability of modern high-speed networks for everyone and access to diverse data. A second main track included improvements to resilience to natural disasters and equipping communities for greater preparedness. Another NSF program SaTC is focused on human-centered approaches to cyber security, addressing complex problems such as disinformation, human abuse and tracking, and child protection against cyber threats. One example of the awarded research was the demonstration that vehicles could be hacked to override their drivetrain. This exploit was a significant issue when uncovered. 

NSF funds billions of dollars in various programs. For example, the infrastructure that we all use includes campus research. One example comprises platforms for advanced wireless research with multiple locations around the U.S. with remotely assessable resources to access testbeds for all kinds of network research. Infrastructure is not just hardware; it's also the data and software resources from which all aspects of CISE research can benefit. NSF has many infrastructure proposals you can contribute to. 

Without people, none of this would be possible. The NSF CISE needs many people from diverse backgrounds to create the next generation of research. Programs such as the BACnet program provide resources to NSF CISE to get people into these programs as well as the NSF is always hiring in critical areas. NSF offers research experiences for students and always accepts proposals and research positions within the NSF.

\section{Lessons Learned}

While much of the research and information we often cite in our academic papers often include research done by the NSF, I rarely learned how these programs and research are funded and created. The vast size and scope of the research NSF work at were unknown to me beforehand, particularly in the area of computer and information science. The CISE area of focus is particularly interesting for our computer science department and knowing the topics that an institution such as NSF is focusing on for the upcoming years, such as post Moore's law design, AI, and sociotechnical issues are vital to know in both our research and any future professional ideas we may encounter. These can inform us to keep in mind any potential impacts our work will have down the road economically or socially. 

\section{Conclusion}

The NSF CISE is not only one of the largest research foundations. It's perhaps the most important. The research done under the funding of the NSF impacts billions of people worldwide for decades to come. In today's presentation, Dr. Margaret Martonosi showed us the extent of NSF CISE research and the critical areas of focus for the future of computer science. Never before has information and technology been more vital to society as a whole, and the NSF CISE is essential to the proper direction and revolution of technology in the future. 

% \appendices
% \section{Proof of the First Zonklar Equation}
% Appendix one text goes here.

% % you can choose not to have a title for an appendix
% % if you want by leaving the argument blank
% \section{}
% Appendix two text goes here.


% use section* for acknowledgment
\section*{Acknowledgment}
The author would like to thank Professor Sajal Das with the Department of Computer Science, Missouri University of Science and Technology and Dr. Margaret Martonosi with the NSF CISE.

% Can use something like this to put references on a page
% by themselves when using endfloat and the captionsoff option.
\ifCLASSOPTIONcaptionsoff
  \newpage
\fi

% \begin{thebibliography}{1}

% \bibitem{IEEEhowto:kopka}
% H.~Kopka and P.~W. Daly, \emph{A Guide to \LaTeX}, 3rd~ed.\hskip 1em plus
%   0.5em minus 0.4em\relax Harlow, England: Addison-Wesley, 1999.

% \end{thebibliography}

% biography section
\begin{IEEEbiographynophoto}{Matthew Whitesides}
  Master's Student at Missouri University of Science and Technology.
\end{IEEEbiographynophoto}

% that's all folks
\end{document}