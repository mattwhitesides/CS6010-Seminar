% options that should be set.
\documentclass[journal,onecolumn]{IEEEtran}

% correct bad hyphenation here
\hyphenation{op-tical net-works semi-conduc-tor}

\begin{document}

%
% paper title
% Titles are generally capitalized except for words such as a, an, and, as,
% at, but, by, for, in, nor, of, on, or, the, to and up, which are usually
% not capitalized unless they are the first or last word of the title.
% Linebreaks \\ can be used within to get better formatting as desired.
% Do not put math or special symbols in the title.
\title{Seminar Talk: ``Constructing a Human Digital Twin'' (Speaker: Professor Diane Cook)}

%
%
% author names and IEEE memberships
% note positions of commas and nonbreaking spaces ( ~ ) LaTeX will not break
% a structure at a ~ so this keeps an author's name from being broken across
% two lines.
% use \thanks{} to gain access to the first footnote area
% a separate \thanks must be used for each paragraph as LaTeX2e's \thanks
% was not built to handle multiple paragraphs
%
\author{Matthew~Whitesides}% <-this % stops a space

% The paper headers
\markboth{Missouri S\&T COMP\_SCI 6010: Seminar, Fall~2021}%
{Shell \MakeLowercase{\textit{et al.}}: Bare Demo of IEEEtran.cls for IEEE Journals}

% make the title area
\maketitle

% As a general rule, do not put math, special symbols or citations
% in the abstract or keywords.
\begin{abstract}
  In today's presentation, professor Diane Cook discusses constructing human digital twins. This potentially life-changing technology can enable the automation of health assessments by creating a vital human body model. Professor Cook takes us through the various challenges, benefits, and outcomes of this digital model-based approach to health care. Precisely the challenges in creating human digital twins in the real world, survey emerging data mining methods, and describe some of the current and future impacts that these technologies can have.
\end{abstract}

% Note that keywords are not normally used for peerreview papers.
% \begin{IEEEkeywords}
% IEEE, IEEEtran, journal, \LaTeX, paper, template.
% \end{IEEEkeywords}

\IEEEpeerreviewmaketitle

\section{Introduction and Background}

\IEEEPARstart{D}{igital} twins are a virtual representation of a real-time digital counterpart of a physical object or process. Typically this serves as a digital model for designing a physical object, i.e., a virtual model of an aircraft that simulates the physical makeup and properties of the object to detect and design issues and simulate various interactions without needing to waste natural world physical resources. Not only for physical modeling, but digital twins greatly aid in continuous iterative improvement by forcing a more data-driven approach to engineering and allowing said data to make informed decisions of changes to the model. 

Using these principles and applying them to the healthcare industry is an exciting proposition. Creating a digital model of not only engineering systems but also actual human health models allows us to take a similar approach to model-based engineering and apply them to human health assessments. The advancement of technologies makes it possible to create personalized digital models for patients. The model's data can be continuously updated with the increasing use of biometric internet of things (IoT) sensors. Not only for an individual patient but the more data we gather on humans in general, the more we can apply discoveries to the entire population. The increasing availability of these sensing platforms and the maturing of data mining methods allowed professor Cook and the team to support building such a replica from longitudinal, passively-sensed data. 

\section{Research Contributions}

To honestly assess an individual, our models need to include genes, genetic markers, social interactions, medical history and physiology, and her physical environment. Luckily the many advancements in IoT and smart devices have enabled passive monitoring in our external environment and biometric markers, an achievable prospect. Various sensors were utilized around a smart home environment to track time-series activity data throughout the day. To achieve this, the models built of the sensor data needed to be sensor agnostic and map the time series data to a specific activity label, much like a fitness tracker guesses your current activity but across various sensors and more particular activities beyond heart rate-based data. In the end, through various environmental, ambient, mobile, and biometric sensors, a complex machine learning model was built with the ultimate goal to map these activities and metrics to health insights. 

\section{Results}

Overall through the collection of continuous sensor data from over forty homes, with an average of five years of data for each individual on an older average age of the patient (85 or older), various transitions in health status were seen in the data. The question becomes, can we build a behavior model from this long-spanning data and then use machine learning to map the behavior onto clinical measures. If so, not only could we detect and assess a patient's clinical status at any given time but predict when events will occur in patients before it becomes obvious. 

Something logical but discovered in measuring the activity data is human behavior. Health is highly variable and unique to each individual, therefore making the act of a digital twin on a specific person important. However, it's essential to generalize the machine learning breakthroughs to apply to as many people as possible. Therefore some key features were combined with the individual such as demographic, existing clinical scores, smart home features, and biometric smartwatch features to build a joint gradient boosting regression model. These features measured jointly greatly improved accuracy on health assessments than each aspect modeled individually. 

Some common challenges include not having access to some features, or for privily reasons, some patients would not want specific ground truths reviewed about themselves through the sensor tracking. These concerns make the generalizable model more critical to apply to as many people as possible with various datasets. To tackle this challenge, they decided to design a time series data-based idea of a generative adversarial network, an approach to generative modeling using deep learning methods, such as a neural network. Neural networks enable complex relations to be extracted from various features using realistic synthetic data in an image domain. Essentially these domains create fake data to generate against to improve upon without a full feature dataset iteratively. 

The ultimate goal is to detect the need for health interventions. We'll want to use this digital twin model to understand the person's health state their activity context to deliver the proper intervention at the right time. Professor Cook and the team noticed that prompts for health interventions were more effective when they were transitioning activities to interact with them, improve their response and improve treatment effectiveness. Utilizing this approach they found much higher rates of interaction with the alerts. 

\section{Lessons Learned}

I think human digital twin research is a fantastic idea and will be vital to the future of healthcare and personal health assessments. This topic is something I have been interested in aspects surrounding the core idea for a while. I have researched biometric IoT devices and utilizing data science/machine learning on biometric data, but that is a small part of the greater focus here of creating a holistic digital twin of a human. Human digital twins have unlimited potential to give insight into decease detection and optimization of human performance in general. The results professor Cook and her team have already found are significant, and as the models mature and more detailed biometric data is available, more insights will reveal themselves. It's not hard to imagine a future where almost any aspect of your health is predictable and available to patients with high accuracy just off these digital models and data mining. Also, tracking this information will lead to more discoveries in treating patients. It will be easy to determine if they are working based on the continuously updating data in the models. 

\section{Conclusion}

Professor Diane Cook showed us the vast potential, various challenges. She surveyed the promising methods in tackling these challenges for the prospect of creating human digital twin models for use in the healthcare industry. Through this research, we've been shown how there is a lot to creating applicable health predictive activity models. The end product contains the extensive data collection through numerous sensors over long periods, a the novel approach's in building the feature models, the creation of a digital twin of an individual while being able to generalize the results to a broader audience, and finally using these results to be able to inform and intervene when an individuals health needs attention.

% \appendices
% \section{Proof of the First Zonklar Equation}
% Appendix one text goes here.

% % you can choose not to have a title for an appendix
% % if you want by leaving the argument blank
% \section{}
% Appendix two text goes here.


% use section* for acknowledgment
\section*{Acknowledgment}
The author would like to thank Professor Sajal Das with the Department of Computer Science, Missouri University of Science and Technology and Professor Diane Cook with Washington State University. .

% Can use something like this to put references on a page
% by themselves when using endfloat and the captionsoff option.
\ifCLASSOPTIONcaptionsoff
  \newpage
\fi

% \begin{thebibliography}{1}

% \bibitem{IEEEhowto:kopka}
% H.~Kopka and P.~W. Daly, \emph{A Guide to \LaTeX}, 3rd~ed.\hskip 1em plus
%   0.5em minus 0.4em\relax Harlow, England: Addison-Wesley, 1999.

% \end{thebibliography}

% biography section
\begin{IEEEbiographynophoto}{Matthew Whitesides}
  Master's Student at Missouri University of Science and Technology.
\end{IEEEbiographynophoto}

% that's all folks
\end{document}